\documentclass[10pt]{article}
\usepackage[usenames]{color} %used for font color
\usepackage{amssymb} %maths
\usepackage{amsmath} %maths
\usepackage[utf8]{inputenc} %useful to type directly diacritic characters
\begin{document}
\[\noindent
{\huge Problem 3} \\

\noindent 
\textbf{My Intuition}

\noindent
If we have data points j = 0 to n, then for their positions i, all data points $$x_{i} = 2^{-i}$$ gives an infinite possible number of points between 0 and 1.
\noindent
For our hypothesis $h_{w}(x)$, looking at the function $y = \sin(wx)$, w relates to the period of our sine wave by $p = \frac{2\pi}{w}$. This means that $y = \sin(\pi x)$ has a period of 2. \\

\noindent
Looking further at the function, we can see that for multiples of 2, our function will pass through the points $ x = 2^{-i}$ at $y = 0$. This means that if we are to correctly classify points with

$$ if \quad y = sin(wx) < 0\quad then\; -1\; or \; False $$
$$ \noindent else\quad 1\; or\; True $$ \\

\noindent
then we have to implement a shift in the period with a small enough magnitude to not incorrectly classify our smallest points. To do this, I looked at the graph of $y = \sin(2\pi x)$ and then implemented an initial shift to the period of 0.01 by changing the equation to $y = \sin((2\pi + 0.01) x)$. Looking in excel at different points $\frac{1}{2}, \frac{1}{4}, \frac{1}{8},$ and $\frac{1}{16}$ classified by different functions based on $y = \sin(wx)$ with $$ w = (\sum\limits_{i=1}^n 2^i) + 2^{n} $$ I found the following pattern.



\]
\end{document}